\chapter*{Introduzione} \label{cap0}
Il rilevamento di persone in scene naturali è molto impegnativo per problemi dovuti alla grande variabilità in apparenza, posa e occlusioni. Negli ultimi anni, anche grazie agli enormi passi avanti fatti in campo tecnologico, in letteratura sono sempre più presenti approcci basati sull'utilizzo delle Reti Neurali Convolutive (CNN). \\
L'approccio leader nelle performance in ImageNet e PASCAL è il R-CNN di Girschik \textit{et al.}\cite{rcnn}, il quale mostra come la localizzazione di persone (e altri oggetti) può essere effettuata dapprima individuando alcune possibili locazioni di persone o oggetti nell'immagine le quali vengono presentate successivamente ad una CNN che effettua la classificazione.\\
Uno dei problemi maggiori nell'utilizzo delle CNN per il rilevamento di persone è che la dimensione dell'input della CNN deve essere fissa mentre in realtà la parte di immagine può avere una proporzione arbitraria.\\
Nell'approccio di Girschik dall'immagine intera vengono estrapolate delle regioni con un metodo bottom-up le quali avranno differenti proporzioni: per adattare la dimensione della regione con la dimensione di input della CNN viene effettuata un'operazione di ridimensionamento, la quale può portare ad immagini molto distorte. Per ovviare a ciò oltre a campioni di addestramento con proporzioni originali vengono utilizzati nella fase di addestramento anche campioni distorti che non esistono in natura, i quali rendono più complesso al sistema l'apprendimento di corrispondenze di uno stesso oggetto. Un ulteriore limite della R-CNN è il metodo utilizzato nella fase preliminare di segmentazione dell'immagine in regioni dal quale dipende la bontà della classificazione finale.\\
Nel seguente lavoro viene proposto un nuovo approccio per il rilevamento di persone in immagini, il quale non prevede nella fase di addestramento l'utilizzo di campioni distorti in proporzioni e una grande variabilità di campioni per lo stesso classificatore. L'approccio utilizza le \textit{Poselets} presentate da Bourdev \textit{et al.}\cite{poselets}: esse sono parti del corpo umano in una particolare posa viste da uno specifico punto di vista. \\
Viene presentato un approccio deep learning che utilizza una CNN come estrattore delle caratteristiche delle poselets che produce un vettore di caratteristiche di sole 256 dimensioni.\\
Il resto dell'elaborato è diviso nel seguente modo. Nel capitolo 1 viene mostrata la struttura della Deep Net e la creazione del training-set, nel capitolo 2 vediamo come funzionano i classificatori Poselet, nel capitolo 3 ci sono dettagli sull'implementazione ed infine nell'ultimo capitolo vengono mostrati vari test.



