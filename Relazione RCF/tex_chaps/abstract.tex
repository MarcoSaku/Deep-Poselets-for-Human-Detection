\chapter*{Abstract} \label{abstract}
L'approccio sviluppato si rivolge al problema del rilevamento di persone in scene naturali utilizzando un metodo basato sul rilevamento di singole parti del corpo, chiamate poselet. Come prima fase si introduce un metodo per la raccolta di milioni di immagini etichettate per ogni tipo di poselet. Queste ultime vengono utilizzate per addestrare una Rete Neurale Convolutiva (CNN) per discriminare i vari tipi di poselet e separarle dallo sfondo. La CNN viene poi utilizzata per ottenere un vettore di caratteristiche per la poselet di 256 dimensioni. Si collezionano delle immagini per ogni tipo di poselet e si addestra un SVM per ciascun tipo di poselet utilizzando il vettore di caratteristiche precedentemente ottenuto. I classificatori SVM vengono utilizzati su patch dell'immagine target per individuare la presenza delle poselets che successivamente vengono combinate per trovare la posizione della persona intera.  